%%%%%%%%%%%%%%%%%%%%%%%%%%%%%%%%%%%%%%%%%
% This document provides a sample senior 
% thesis proposal template for use
% by Allegheny's Computer Science majors.
%
% This template was adopted from Jeremie Gillet
% Ref: https://github.com/oist/LaTeX-templates
%
% Author: Janyl Jumadinova
% Last Updated: September 10, 2019
%
%%%%%%%%%%%%%%%%%%%%%%%%%%%%%%%%%%%%%%%%%

%----------------------------------------------------------------------------------------
%	PACKAGES AND OTHER DOCUMENT CONFIGURATIONS
%----------------------------------------------------------------------------------------

\documentclass[12pt,oneside]{book} % 12 pt font, one-sided book style
\usepackage[a4paper, includehead, headheight=0.6cm, inner=3cm ,outer=2.5cm, top=2.5 cm, bottom=2.5cm]{geometry}  % Changing size of document
\usepackage[english]{babel} % The document is in English
\usepackage[utf8]{inputenc} % UTF8 encoding
\usepackage[T1]{fontenc} % Font encoding

\usepackage{graphicx} % For including images
\graphicspath{{./images/}} % Specifies the directory where images are stored

\usepackage{longtable} % tables that can span several pages
\usepackage[bf]{caption} % caption: FIG in bold
\usepackage{fancyhdr} % For the headers

\newcommand{\numberedchapter}{ % Preparation for numbered chapters
	\cleardoublepage % To make sure the previous headers are passed
	\fancyhead[RE]{{\bfseries \leftmark}}% Headers for left pages
	\fancyhead[LO]{{\bfseries \rightmark}}}% Headers for right pages
\newcommand{\unnumberedchapter}[1]{ % Preparation for unnumbered chapters
	\cleardoublepage % To make sure the previous headers are passed
	\addcontentsline{toc}{chapter}{#1} % Also adds the chapter name to the Contents
	\fancyhead[RE]{{\bfseries #1}} % Headers for left pages
	\fancyhead[LO]{}}%Headers for right pages

\usepackage{emptypage} % No headers on an empty page

\usepackage{eso-pic} % For the background picture on the title page
\newcommand\BackgroundPic{%
\put(0,-120){%
\parbox[b][\paperheight]{\paperwidth}{%
\vfill
\centering
\includegraphics[width=5in]{images/logo}%
\vfill
}}}

\usepackage{hyperref} % Adds clickable links at references

%----------------------------------------------------------------------------------------
%	ADD YOUR CUSTOM VALUES, COMMANDS AND PACKAGES
%----------------------------------------------------------------------------------------

% Open preamble/mydefinitions.tex and enter some values (name, thesis title...) 
% and include your own custom LaTeX functions and packages

%----------------------------------------------------------------------------------------
% values for the proposal
%----------------------------------------------------------------------------------------

\newcommand{\name}{Aravind Mohan} % Author name
\newcommand{\thesistitle}{Proposal Title} % Title of the thesis
\newcommand{\submissiondate}{September 14, 2020} % Submission date "Month, date year"
\newcommand{\supervisor}{First Reader} % First reader's name
\newcommand{\cosupervisor}{Second Reader} % Second reader's name


%----------------------------------------------------------------------------------------
%	BIBLIOGRAPHY STYLE 
%----------------------------------------------------------------------------------------


\bibliographystyle{acm}

%----------------------------------------------------------------------------------------
%	YOUR PACKAGES (be careful of package interaction)
%----------------------------------------------------------------------------------------

\usepackage{amsthm,amsmath,amssymb,amsfonts,bbm}% Math symbols

%----------------------------------------------------------------------------------------
%	YOUR DEFINITIONS AND COMMANDS
%----------------------------------------------------------------------------------------

% New Commands
\newcommand{\bea}{\begin{eqnarray}} % Shortcut for equation arrays
\newcommand{\eea}{\end{eqnarray}}
\newcommand{\e}[1]{\times 10^{#1}}  % Powers of 10 notation


\begin{document}

%----------------------------------------------------------------------------------------
%	TITLE PAGE
%----------------------------------------------------------------------------------------

\pagestyle{empty} % No page numbers
\frontmatter % Use roman page numbering style (i, ii, iii, iv...) for the preamble pages

\begin{titlepage}
\AddToShipoutPicture*{\BackgroundPic}
\begin{center}
\vfill
{\large \scshape Allegheny College \\ Department of Computer Science }\\[1.4cm]
{\Large Senior Thesis Proposal}\\[0.5cm]
\rule{\textwidth}{1.5pt}\\[0cm]
{\huge \bfseries \thesistitle \par \ }\\[-0.5cm]
\rule{\textwidth}{1.5pt}\\[2.5cm]
\hfill  by\\[1cm]
\hfill  {\large \bfseries\name}\\
\vfill
{\hfill \large Project Supervisor: \textbf{\supervisor}} \\ 
\ifx\cosupervisor\undefined\else{\hfill \large Co-Supervisor: \textbf{\cosupervisor}} \\ \fi
\vspace{1cm}
\hfill  \submissiondate
\end{center}
\end{titlepage}

%----------------------------------------------------------------------------------------
%	PREAMBLE PAGES (comment out unnecessary pages)
%----------------------------------------------------------------------------------------

\pagestyle{fancy} % Changes the headers
\fancyhf{}% Clears header and footer
\fancyhead[RO,LE]{\thepage} % page number on the outside of headers

\unnumberedchapter{Abstract} 
\chapter*{Abstract} 
\subsection*{\thesistitle}

Provide a concise summary of your proposed research of approximately 250 words. 
Remember that the abstract is {\it not\/} an introduction, it is a {\it summary\/} of the entire document.
It makes sense to wait to write the abstract until the rest of the document has been written.

\addtocontents{toc}{\vspace{2em}} % Add a gap in the Contents, for aesthetics
\mainmatter % Begin numeric (1,2,3...) page numbering


%----------------------------------------------------------------------------------------
%	DELETE TEXT
%----------------------------------------------------------------------------------------

\section*{Template Overview}

It is part of the student's training in research to prepare a concise, rigorous, and scholarly thesis proposal and present it in the correct format. There is no strict length requirement for the senior thesis proposal. It is anticipated that most students will need about twenty pages of text to adequately explain the motivation and goals of their project, review the relevant literature, and describe the feasibility of the proposed work. However, concise proposals are also encouraged. 

You should first modify the documents in the preamble, things that appear before the main text as detailed below. 

\textbf{Front page}: use the one provided in this template, after changing the values like names in the file \texttt{preamble/mydefinitions.tex}.

\textbf{Abstract}: There should be a single paragraph of about 250 words, which concisely summarizes the entire proposal, written in the file \texttt{preamble/abstract.tex}.

The main text of the proposal should be store in the ``SeniorThesisProposal.text'' document. The following descriptions are sections that must be included in the proposal.


%----------------------------------------------------------------------------------------
% STOP DELETE
%----------------------------------------------------------------------------------------

\section*{Introduction}
\label{sec:introduction}

This section should include a statement of the problem, the overall aims, and the background motivating your research. Whenever possible, you should use one or more concrete examples
and technical diagrams.

\section*{Related Work}
\label{sec:relatedwork}

This section should include the review of relevant existing work (literature review). The literature review should be a concise, scholarly review of the literature explaining the background to the proposed research. The review should provide the context for the aims of the proposed research in relation to existing work on the topic. 

\section*{Method of Approach}
\label{sec:method}

This section should answer the ``how'' question - how will you perform the proposed research. 
It should also describe the feasibility study you have conducted to demonstrate that your project is feasible. Use technical diagrams, equations, algorithms, and paragraphs of text to
describe the research that you intend to complete. Be sure to number all figures and tables and to explicitly refer to them in your text.

\section*{Evaluation Strategy}
\label{sec:evaluate}

This section should explain what steps you will take to evaluate your proposed method. If you intend
to conduct experiments, then you must clearly define your evaluation metrics.

\section*{Research Schedule}
\label{sec:plan}

 This section identify the main phases and tasks of your research project and set deadlines
for when you will be able to complete each of these items. Please remember that you should aim to complete the project before the middle of March.

\section*{Conclusion}
\label{sec:conclusion}

Provide a summary of your proposed research and suggest the impact that it may
have on the discipline of computer science. If possible, you may also suggest
some areas for future research.


 
%----------------------------------------------------------------------------------------
%	BIBLIOGRAPHY
%----------------------------------------------------------------------------------------

\addtocontents{toc}{\vspace{2em}} % Add a gap in the Contents, for aesthetics
\unnumberedchapter{Bibliography} % Title of the unnumbered chapter
\textbf{Bibliography}: The bibliography should include all references cited in the text (as \cite{dasgupta2015comrade}) and it should not include references that have not been cited. ACM referencing style should be used when preparing the bibliography. We recommend using BibTeX or BibLaTeX and using the file \texttt{preamble/bibliography.bib}.
\bibliography{preamble/bibliography} % The references information are stored in the file named "bibliography.bib"


\end{document}  